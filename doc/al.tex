\chapter{Algoritmia}

\section{Búsqueda exhaustiva, fuerza bruta o backtracking, como lo quieras llamar}
Buscar todo/casi todo el espacio de soluciones, y hasta encontrar una solución. Es todo lo lento
que puede ser, pero no es mala idea intentarlo si no se ocurre otra idea.
\lst{brute.cpp}

\section{Divide y vencerás}
Consiste en seguir tres sencillos pasos: dividir el problema en sub-problemas
(como por la mitad más menos), encontrar soluciones a esos problemas más
pequeños (que será más fácil, porque son más pequeños) y combinar las soluciones
de los subproblemas. Normalmente consiste en utilizar una estructura que haga esto
(montículos, bBST, etc.) o en hacer \textbf{búsqueda binaria de la solución}.
\lst{bsta.cpp}


\section{Voraz, greedy}

\section{Programación dinámica}
\subsection{LIS}
