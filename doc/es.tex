\chapter{Estructuras}
\section{Listas, Pilas, colas, sets, hashmaps, heaps}
Todas las estructuras de datos básicas que se pueden encontrar en la STL de C++.

\begin{lstlisting}
vector<int> lista;                      // Suele venir bien inicializada
stack<int> pila;                        // LIFO
queue<int> cola;                        // FIFO
priority_queue<int> cola_prioridad;     // La cabeza es el mayor elemento
unordered_set<int> conjunto;            // Conjunto de elementos únicos
unordered_map<string, int> hashmap;     // Mapa de clave-valor, sin orden

// No hay clase para el heap, pero se puede usar un vector
vector<int> H;
is_heap(H.begin(), H.end());    // Comprueba si es un heap
make_heap(H.begin(), H.end());  // Crea un heap a partir de un vector
push_heap(H.begin(), H.end());  // Incluye el último elemento en el heap
pop_heap(H.begin(), H.end());   // Mueve el mayor elemento al final
sort_heap(H.begin(), H.end());  // Convierte el heap en un vector ordenado
\end{lstlisting}

\section{Conjuntos disjuntos}
Modela conjuntos disjuntos. Útil para encontrar componentes conexos en grafos.

\lst{ds/ufds.cpp}

\section{Grafos}
Los grafos se pueden representar de varias formas, pero la más común es mediante
una lista de adyacencia (es la que se usa en todos). También hay veces que no
hace falta una estructura específica para los grafos. Ordenados de más a menos
uso:
\lst{ds/graph.cpp}

\subsection{BFS y DFS}

\subsection{Dijkstra}
\subsection{MST}
\subsection{Bipartitos}
\subsection{Max flow}
\subsection{Árboles binaros}
\subsection{Árboles de segmentos}
\subsection{Árbol de Fenwick}
