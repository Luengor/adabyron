\chapter{Mates}
\section{Primos}
Unos cuantos primos muy grandes: $1e9 + 7$, $1e9 + 9$, $1e9 + 21$.

\subsection{Eratóstenes}
\lst{alg/criba.cpp}

\subsection{Cosas con factores}
\lst[6]{alg/fac_primos.cpp}

\section{Aritmética modular}
\subsection{EGCD}
\lst{alg/mod_inv.cpp}

% Teorema chino del resto?

\section{Combinatoria}
\subsection{Números catalanes}
Los números catalanes cuentan un montón de cosas, como:
\begin{itemize}
    \item $Cat(n)$ es el número de árboles binarios con $n$ nodos.

    \item $Cat(n)$ el número de expresiones de paréntesis bien formadas con $n$
    pares de paréntesis.

    \item $Cat(n)$ el número de formas de dividir un polígono convexo de $n+2$
    lados en triángulos.

    \item $Cat(n)$ el número de caminos de $n$ pasos que no suben por encima de
    la diagonal en un plano.
\end{itemize}

Números catalanes, fibonacci y coeficientes binomiales
\lst{alg/fib_cat.cpp}

\subsection{Nth Permutation con Factoradic}
El orden de las permutaciones se puede calcular con un sistema factorádico, que
es una representación de números enteros en base factorial. Cada dígito del
número factorádico representa la cantidad de veces que se usa cada
factorial en la descomposición del número. (Por ejemplo, el número 1010 en
factorádico es 1*3! + 0*2! + 1*1! + 0*0! = 6 + 0 + 1 + 0 = 7). Se puede
utilizar para encontrar la $k$-ésima permutación de un conjunto de elementos
sin necesidad de generar todas las permutaciones (muy rápido con un árbol de
fenwick).

\lst{alg/factoradic.cpp}

% orden de permutaciones con sistemas factorádicos
\section{Matrices}
\subsection{Ecuaciones}
Se puede resolver un sistema de ecuaciones lineales utilizando el método de Gauss
a partir de la matriz aumentada del sistema.
\begin{align}
a_{11} x_1 + a_{12} x_2 + &\dots + a_{1m} x_m = b_1 \\
a_{21} x_1 + a_{22} x_2 + &\dots + a_{2m} x_m = b_2\\
&\vdots \\
a_{n1} x_1 + a_{n2} x_2 + &\dots + a_{nm} x_m = b_n
\end{align}

\lst{alg/gauss.cpp}

\subsection{Rank}
El rango de una matriz es el número máximo de columnas linealmente
independientes. Se puede calcular utilizando el método de eliminación de Gauss.
\lst{alg/mat_rank.cpp}

\subsection{Determinante}
No te lo vas a creer, pero el determinante de una matriz se puede calcular
utilizando un método que tiene un nombre que se parece a Mouse.
\lst{alg/det_gauss.cpp}

\subsection{Multiplicación}
\lst{alg/mat_mul.cpp}

